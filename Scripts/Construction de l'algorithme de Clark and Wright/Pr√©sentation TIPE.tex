\documentclass[handout]{beamer}
%%% Pour le français %%%
\usepackage[utf8]{inputenc}
\usepackage[T1]{fontenc}
\usepackage[frenchb]{babel}
%%%%%%%%%%%%%%%%%%%%%%%%
% \usepackage{fancyhdr} % En-tête et pied de page personnalisés
% \usepackage{listings} % Pour du beau code coloré
%%%% Pour des maths %%%%
% \usepackage{mathtools} % Pour pleins de commandes avec des maths
% \usepackage{amssymb} % Pour les symboles
% \usepackage{amstext} % Pour utiliser \text
% \usepackage{mathrsfs} % 3 fonts pour les 26 lettres
% \usepackage{amsthm} % Custom des theoremes
% \usepackage{tikz} % Pour des graphiques
%%%%%%%%%%%%%%%%%%%%%%%%
% \usepackage{layout} % Pour afficher le gabarit de mise en page
% \usepackage{geometry} % Pour régler les marges
% \usepackage{setspace} % Pour modifier l'interligne
% \usepackage{ulem} % Pour souligner et barrer du texte
%%% Pour des polices %%%
% \usepackage{bookman}
% \usepackage{charter}
% \usepackage{newcent}
% \usepackage{lmodern}
% \usepackage{mathpazo}
% \usepackage{mathptmx}
%%%%%%%%%%%%%%%%%%%%%%%%
% \usepackage{url} % Pour citer des urls
% \usepackage{graphicx} % Pour travailler sur des images
% \usepackage{color} % Pour manipuler les couleurs et colorer le texte

\title[TIPE — Optimisation des transports]{TIPE \\ Optimisation des transports}
\author{ROCHER Kilian — WILLEM Logan}
\date{2020 - 2021}

\mode<presentation>

\useoutertheme[footline=authortitle]{miniframes}
\useinnertheme{circles}
\usecolortheme{whale}
\usecolortheme{orchid}

\definecolor{beamer@blendedblue}{rgb}{0.137,0.466,0.741}
\definecolor{titleColor}{RGB}{102,153,255}
\definecolor{textColor}{RGB}{60,60,60}

\setbeamercolor{titlelike}{bg=titleColor}
\setbeamercolor{titlelike}{parent=structure}
\setbeamercolor{frametitle}{fg=black}
\setbeamercolor{title}{fg=black}
\setbeamercolor{item}{fg=black}
\setbeamercolor{normal text}{fg=textColor}
\setbeamertemplate{background canvas}[vertical shading][top=cyan!7!white,bottom=cyan!2!white]
\setbeamertemplate{blocks}[rounded][shadow=true]

\begin{document}
	\begin{frame}[plain]
		\maketitle
	\end{frame}

	\begin{frame}[plain]
		\tableofcontents
	\end{frame}

	\section{Situation du problème}
	
	\subsection{Qu'est ce que l'optimisation des transports}
	
	\begin{frame}
		Definition et enjeux de l'optimisation des transports
	\end{frame}
	
	\subsection{Quelques exemples}
	
	\begin{frame}
		Quelques exemples d'optimisation de transport avec images explicatives
	\end{frame}
	
	\section{La tournée des véhicules}
	
	\subsection{Contexte}
	
	\begin{frame}
		Explication approfondie de la tournée des véhicules
	\end{frame}

	\subsection{Variantes}
	
	\begin{frame}
		Quelques variantes intéréssantes de la tournée des véhicules
	\end{frame}
	
	\section{Résolution}

	\subsection{L'algorithme de Clark \& Wright}

	\begin{frame}
		Explication de l'algorithme de Clark and Wright avec démonstration
		
		\begin{proof}
			démo à insérée
		\end{proof}
	
	\end{frame}

	\subsection{Insufisance de l'algorithme}

	\begin{frame}
		Résultats de l'algorithme et présentation de ses faiblesses / insufisances
	\end{frame}

	\subsection{Amélioration : le 2-opt}

	\begin{frame}
		Principe du 2-opt et résultats combinés
	\end{frame}

\end{document}