\documentclass[14pt]{article}
%%% Pour le français %%%
\usepackage[utf8]{inputenc}
\usepackage[T1]{fontenc}
\usepackage[french]{babel}
%%%%%%%%%%%%%%%%%%%%%%%%
\usepackage{fancyhdr} % En-tête et pied de page personnalisés
\usepackage{listings} % Pour du beau code coloré
%%%% Pour des maths %%%%
% \usepackage{mathtools} % Pour pleins de commandes avec des maths
 \usepackage{amsmath}
 \usepackage{amssymb} % Pour les symboles
 \usepackage{amstext} % Pour utiliser \text
 \usepackage{mathrsfs} % 3 fonts pour les 26 lettres
 \usepackage{amsthm} % Custom des theoremes
 \usepackage{tikz} % Pour des graphiques
%%%%%%%%%%%%%%%%%%%%%%%%
% \usepackage{layout} % Pour afficher le gabarit de mise en page
% \usepackage{geometry} % Pour régler les marges
% \usepackage{setspace} % Pour modifier l'interligne
% \usepackage{ulem} % Pour souligner et barrer du texte
%%% Pour des polices %%%
% \usepackage{bookman}
% \usepackage{charter}
% \usepackage{newcent}
% \usepackage{lmodern}
% \usepackage{mathpazo}
% \usepackage{mathptmx}
%%%%%%%%%%%%%%%%%%%%%%%%
% \usepackage{url} % Pour citer des urls
% \usepackage{graphicx} % Pour travailler sur des images
% \usepackage{color} % Pour manipuler les couleurs et colorer le texte
\usepackage{enumitem}

\title{Principe de l'algorithme de Clark and Wright}
\author{}
\date{}

\pagestyle{fancy}

\chead{}
\lhead{}
\rhead{MPSI3}
\lfoot{}
\cfoot{}
\rfoot{\thepage}


\begin{document}

\maketitle

\pagebreak
\noindent Le principe de l'algorithme de Clark and Wright repose sur le principe suivant :
\\ \newline
\underline{\textbf{Données :}}
\begin{itemize}[label=•]
    \item $D$ : Dépot de coordonnées $(0,0)$
    \item Une famille de points \((i_1,...,i_k) \in {(\lbrack-100,100\rbrack^2)}^k\) pour un certain $k \in \lbrack2,+\infty\lbrack$
    \item Une fonction $d$ qui calcule la distance entre deux points.
\end{itemize}
\ \newline
On introduit la fonction $s$ qui calcule le gain après raccord de deux routes. Celle-ci calcule, pour deux points $i$ et $j$, la différence entre le chemin \\ $D-i-D$ + $D-j-D$ qui vaut donc $2d(D,i) + 2d(j,D)$ au chemin $D-i-j-D$ donc la distance vaut $d(D,i) + d(i,j) + d(j,D)$

\begin{align*}
    s(i,j) &= 2d(D,i) + 2d(j,D) - \lbrack d(D,i) + d(i,j) + d(j,D)\rbrack \\
    s(i,j) &= d(D,i) + d(j,D) - d(i,j)
\end{align*}


\end{document}