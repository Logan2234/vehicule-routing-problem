\documentclass{article}
\usepackage{geometry}
\geometry{a4paper}
\usepackage{graphicx}
\usepackage{amssymb}
\usepackage{amsmath}
\usepackage{amsthm}
\usepackage{empheq}
\usepackage{mdframed}
\usepackage{booktabs}
\usepackage{lipsum}
\usepackage{graphicx}
\usepackage{color}
\usepackage{psfrag}
\usepackage{pgfplots}
\usepackage{bm}
% Additional Settings Here

\title{Étapes pour le construction du "saving algorithm", ou l'algorithme de Clarke and Wright.}
\date{}

\begin{document}
\maketitle
\subsection*{Étape 1 : Calcul des bénéfices}
\begin{itemize}
    \item Calcul des bénéfices ${s_{ij}=c_{i0}+c_{0j}-c_{ij}}$ pour ${i,j=1,…,n}$ et ${i \neq j}$.
    \item Créér ${n}$ routes ${(0,i,0)}$ pour ${i=1,…,n}$.
    \item Trier les bénéfices dans l'ordre décroissant
\end{itemize}
\subsection*{Étape 2 : Meilleure fusion possible}
En commencant par le début de la liste des bénéfices créée, éxecuter ce qui suit :
\begin{itemize}
    \item Etant donné un bénéfice $s_{ij}$, déterminer si il existe deux routes pouvant être fusionnées : 
    \begin{itemize}
        \item Une commençant par $(0,j)$
        \item One ending with $(i,0)$
    \end{itemize}
    \item Combiner ces deux routes en supprimant $(0,j)$ et $(i,0)$ en introduisant $(i,j)$
\end{itemize}
 
\end{document}